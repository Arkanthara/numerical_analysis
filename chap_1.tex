\documentclass{article}
\usepackage{amsfonts}
\newcommand{\R}{\mathbb{R}}
\newcommand{\N}{\mathbb{N}}

\title{Chapitre 1: Interpolation polynomiale}

\begin{document}

\maketitle

\section{Les polynômes de Lagrange et la formule de Newton}
\subsection{Théorème}
Soient les n + 1 points $(x_0, y_0), . . . , (x_n , y_n )$ où les $x_i$ sont distincts. Alors, il
existe un polynôme unique $p_n$ de degré $\leq n$, appelé le polynôme d’interpolation, tel que
$$p_n (x_i ) = y_i\ pour\ i = 0, 1, . . . , n$$.

\subsection{Définition (différences divisées)}
Soient les couples $(x_i, y_i)$ avec chaque $x_i$ distinct pour $i = 0, \dots, n$. On définit:
$$\delta y[x_i, x_{i+1}]\ = \ \frac{y_{i + 1} - y_i}{x_{i+1} - x_i}$$
et pour $k = 2, 3, \dots$, on a:
$$\delta^k y[x_i, x_{i + 1}]\ = \ \frac{\delta^{k - 1} y[x_{i + 1}, \dots, x_{i + k}]- \delta^{k-1} y[x_i, \dots, x_{i + k - 1}]}{x_{i + k} - x_i}$$
À modifier ????
On note $\delta^0 y[x_i] = y[x_i] = y_i$
\subsection{Théorème (Formule de Newton, 1669)}
Le polyôme de Newton est défini par:
$$p_n(x) = c_0 + c_1(x-x_0) + \dots + c_n(x - x_0)\dots (x-x_{n - 1})$$
avec:
$$c_k = \delta^k y[x_0, \dots, x_k]$$
Il est de degré $\leq$ n et passe par les points $(x_0, y_0), \dots, (x_n, y_n)$ où les $x_i$ sont distinct.

\subsection{Lemme}
Soit $f: [a, b] \rightarrow \R$ une fonction n fois dérivable et soit $y_i = f(x_i)$ pour $x_0, \dots, x_n \in [a, b]$ distincts.

Alors $\exists \zeta \in [a, b]$ tel que:
$$\delta^n y[x_0, \dots, x_n] = \frac{f^{(n)}(\zeta)}{n!}$$

\subsection{Théorème}
Soit $f: [a, b] \rightarrow \R$ une fonction $n+1$ fois dérivable et $p(x)$ le polynôme d'interpolation de degré $\leq$ n et passant par les points $(x_0, f(x_0)), \dots, (x_n, f(x_n))$
($x_i \in [a, b]$ et distincts).

Alors $\forall x$, $\exists \zeta$ dépendant de $x$ tel que:
$$f(x) - p(x) = (x - x_0)\dots(x - x_n)\frac{f^{(n + 1)}(\zeta)}{(n + 1)!}$$

\subsection{Définition (Polynômes de Chebyshev)}
Les polynômes de Chebyshev sont définis $\forall n \in \N$ par:
$$T_n(x) = cos(n * arcos(x))\ \forall x \in [-1, 1]$$

\subsection{Propriétés (Polynômes de Chebyshev)}
\begin{itemize}
    \item $T_n$ est un polynôme de degré $n$ et pour $n \geq 1$ de la forme: $T_n(x)=2^{n-1}x^n + \dots$
    \item $T_n$ satisfait la récurrence:
    \begin{itemize}
        \item $T_0(x) = 0$
        \item $T_1(x) = x$
        \item $T_{n + 1}(x) = 2xT_n(x) - T_{n -1}(x)$
    \end{itemize}
    \item $|T_n(x)| \leq 1$ pour $x \in [-1, 1]$
    \item $T_n(cos\frac{(2k + 1)\pi}{2n}) = 0$ pour $k = 0, \dots, n - 1$
    \item $T_n(cos\frac{k\pi}{n}) = (-1)^k$ pour $k = 0, \dots, n$
\end{itemize}

\subsection{Lemme}
Soit $q(x) = 2^{n - 1}x^n + b_{n-1}x^{n - 1} + \dots + b_0$ tel que $q(x) \neq T_n(x)$

Alors $\forall x \in [-1, 1]$, on a:

$$max(|q(x)|) > max(|T_n(x)|) = 1$$

\subsection{Théorème}
L'expression $|(x - x_0)\dots(x - x_n)|$ est minimale pour $x \in [a, b]$ pour toutes les divisions $a \leq x_0 < \dots < x_n \leq b$
si et seulement si $\forall k = 0, \dots, n$
$$x_k = \frac{b + a}{2} + \frac{b - a}{2}cos\frac{(2k + 1)\pi}{2n + 2}$$

\subsection{Théorème}
Soit $f(x)$ une fois continuement dérivable sur $[a,b]$ et $p_n$ le polynôme d'interpolation passant par $(x_0, f(x_0)), \dots, (x_n, f(x_n))$
et respectant $\forall k = 0, \dots, n$:
$$x_k = \frac{b + a}{2} + \frac{b - a}{2}cos\frac{(2k + 1)\pi}{2n + 2}$$
Alors pour $ n \rightarrow \infty$
$$\max_{x \in [a,b]}|f(x) - p(x)| \rightarrow 0$$

\subsection{Théorème}
Soient les points $x_0, \dots, x_n$ tous distincts et les points $y_0, \dots, y_n$ et $z_0, \dots, z_n$.

Alors il existe un unique polynôme $p_{2n + 1}$ de degré $\leq 2n + 1$, appelé le polynôme d'interpolation d'Hermite tel que $\forall i = 0, \dots, n$:
\begin{itemize}
    \item $p_{2n + 1}(x_i)  = y_i$
    \item $p_{2n + 1 }^{'}(x_i) = z_i$
\end{itemize}

\subsection{Théorème}
Soit $f: [a, b] \rightarrow \R$ $2n + 2$ fois dérivable et $p(x)$ le polynôme d'interpolation d'Hermite.

Alors $\forall x \in [a, b]$, $\exists \zeta $ tel que:
$$f(x) - p(x) = (x - x_0)^2\dots(x - x_n)^2 \frac{f^{(2n + 2)}(\zeta)}{(2n + 2)!}$$

\section{Méthodes}
À compléter !!!!

\end{document}